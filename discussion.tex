%This is an example for a chapter, additional chapter can be added in the skeleton-thesis
%To generate the final document run latex, build and quick build commands on the skeleton-thesis file not this one.
%This is chapter 2, the default skeleton thesis expects 2 chapters
\chapter{Discussion}
\label{sec:discussion}
%The approach utilized in this study is not a realistic replcation of user use since not all activities are meant to be executed by the user and are supposed to be called by another activity.  Consequently, it results in a significantly high number of exceptions.  For purposes of this study, completing a baseline evaluation facilitates the ability to identify the behavioral chages resulting from permission removal.

The evaluation shows that the rate of failure varies with the permission.  In this test, removing permissions only caused 36 (5.3\%) of the 685 applications to crash.   In the best cases, removing \texttt{CAMERA}, \texttt{RECORD\_AUDIO} and \texttt{WRITE\_SMS} respectively, never caused crashes due to permission removal.  While the random sample did not result in any \texttt{SecurityExceptions} with the removal of \texttt{WRITE\_SMS}, it is probable that this is due to the API call being triggered by UI elements deep in the UI structure.  Furthermore, the results from the random sample showed that, if an application's sole use of the \texttt{INTERNET} permission is for advertising, \texttt{SecurityExceptions} caused by its removal are likely to be handled by the ad library.  If the application makes use of network functionality on its own, as in the manually selected test case, the application may crash with a \texttt{SecurityException}. Lastly, it was determined that removing the \texttt{READ\_CONTACTS} and \texttt{ACCESS\_FINE\_LOCATION} permissions had the greatest impact causing 19 and 12 applications to crash respectively.      

\paragraph{\bfseries Wrapper Code}
The manual inspection shows that many wrapper code pieces handle the lack of permissions gracefully.  Wrapper code includes both system services such as Audio Manager and Camera, as well as developer-defined libraries like those used for advertising and the ScoreLoop SDK.  System services provide abstraction between \texttt{getHostByName()} in the developer's code and the hardware.  For example, an application trying to use the device's camera can fail gracefully in the event the device has no camera.  This is also true of permissions revocation.  When the relevant permission is removed, the service acts as if the hardware it manages does not exist.  Third-party libraries can accomplish the same feat by catching the exception on behalf of the developer.  Google AdMob, for example, will display a message to the user running an application without the necessary permissions to obtain the ads (See \ref{fig:removing}).  Notably, it will allow the application to continue to function without advertising.  

While these wrapper code pieces make it easier for the user to remove the permission without causing crashes, they make it harder for the developer to detect the lack of declared permissions programmaticaly.  To address this, developers could make their own API calls to force the exception to be thrown if they wanted to take their own action on permissions revocation as is done by the Scoreloop SDK. 

\paragraph{\bfseries Protected URI}
In situations where an application accesses protected URIs directly (not through wrapper code), it was verified that the application throws \texttt{SecurityExceptions} when it accesses the URIs.  In this case, the developer can easily catch and handle the exception appropriately.  However, this does require explicit modification of existing code, and, without it, users cannot easily remove access without their applications crashing.  

\paragraph{\bfseries Protected API}
The use of protected API calls can result in similar behavior to that of protected URIs.  It is important to note, however, that the exception-throwing behavior of each method may change between Android API revisions.  In previous iterations of the testing, it was observed that the \texttt{getHostByName()} method, used to perform DNS lookups, failed gracefully under Android 4.0.4 (API 15) and returned an error value; applications behaved as if there was no Internet connectivity.  On Android 4.2.2 (API 17), the method generates a \texttt{SecurityExceptions} when the \texttt{INTERNET} permission is removed causing applications to crash.  Therefore, to ensure maximum compatibility, developers should always catch \texttt{SecurityException} when using these APIs.

