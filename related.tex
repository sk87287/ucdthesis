%This is an example for a chapter, additional chapter can be added in the skeleton-thesis
%To generate the final document run latex, build and quick build commands on the skeleton-thesis file not this one.
%This is chapter 2, the default skeleton thesis expects 2 chapters
\chapter{Related Work}
\label{sec:related}
Recently, there have been a number of academic and non-academic works in the area of Android permissions. The work that most closely relates is ~\cite{Hornyack:2011:TAD:2046707.2046780}. In their work, they attempt to analyze the effects of returning fake data to the sensitive API calls who's permissions they want to restrict. The behavioral changes are analyzed by using image comparison techniques. The work performed in this study differs from their's 
in four significant ways. Firstly, the removal of permissions is tested without modifying the Android framework, giving a more realistic  experiment. Secondly, all of the testing is performed autonomously on emulators and does not require human generated application specific scripts, resulting in a scalable testing method. Thirdly,  the samples are used completely random where as their's are significantly skewed towards over permissioned applications which results in an admittedly overestimate of the side effects. Finally, the applications exception behavior is observed instead of using image analysis to detect changes when permissions are removed.  The overall difference is this study attempts to quantify the behavior of an application when a permission is removed, where as ~\cite{Hornyack:2011:TAD:2046707.2046780} focuses on analyzing the effects of using fake instead of removing permissions from applications.   

In ~\cite{Rastogi:2013:AAS:2435349.2435379}, they also performed simlar work.  Like this study, they used UI introspection; however, they utilized humans to record input while the approach utilized in this study generates dynamic input.  In addition, they focus on network traffic while the focus of this research is on an application's on device behavior.  

%A study performed by a team at UC Berkeley found that a large percentage of Android apps were requesting more permissions than they actually used. At the time, Google did not have a complete mapping of Android API calls to the permissions they use; therefore, the team created their own. We made use of this resource often throughout our project~\cite{Felt}. Furthermore, the paper includes a detailed description of the Android permissions mechanism which we will not attempt to duplicate here.

Some third party Android distributions, such as CyanogenMod~\cite{Demers} and MockDroid~\cite{Beresford:2011:MTP:2184489.2184500}, also modify the Android OS to enable permission removal. Like ~\cite{Hornyack:2011:TAD:2046707.2046780}, MockDroid provides fake data to API calls where permissions are being removed. CyanogenMod, however, enables the user to actually revoke permissions. While CyanogenMod essentially accomplishes the changes proposed by this study, the developers of CyanogenMod state concerns about applications failing as a result of permission removal. In addition, these third party distributions are cumbersome to many non-technical users due to the fact that they require you to re-flash the firmware of your device thus voiding its warranty. As a workaround, others have developed applications that allow users to implement the security model described earlier. With the exception of the latest to hit the market, Plop, they all require “rooting”. Among these, LBE Privacy Guard and PDroid provide fake data to handle security exceptions while Plop and Permission Denied do not~\cite{Hoffman}. Using fake data theoretically decreases the number of exceptions incurred, however, it is still unable to handle all of them, i.e. writing to external storage. Ultimately, revoking permissions or providing fake data, regardless of which method one uses, can lead to applications crashing and unexpected behavior. In the end, the better solution is for Google to alter the security model of the Android OS and for developers to handle the exceptions. 
