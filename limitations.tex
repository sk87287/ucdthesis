%This is an example for a chapter, additional chapter can be added in the skeleton-thesis
%To generate the final document run latex, build and quick build commands on the skeleton-thesis file not this one.
%This is chapter 2, the default skeleton thesis expects 2 chapters
\chapter{Limitations and Future Work}
\label{sec:future}

One of the limitations of this work, and dynamic analysis in general, is the issue of code coverage.  The approach presented relies on exercising the application through its UI, and is unable to guarantee that all sensitive APIs will be executed calls within an application. Even if one were able to explore all elements of the UI as defined in the APK's resources, chances are this would still not result in complete coverage.  Given that the presented approach is based solely on dynamic analysis, it is currently not possible to obtain the list of all the API calls an application is able to make. Consequently, code coverage cannot be estimated.  Applications whose execution relies on a WebView present further challenges, as their contents are not visible to the UI introspection techniques used and would require a special coverage calculation scheme.  In addition, \toolname~does not attempt to handle native code within applications.  While no native code was discovered in the random application samples, this complicates some of the dynamic analysis techniques, and would require its coverage to be measured differently as well.

In future iterations of \toolname, the intent is to implement methods to increase and accurately determine the extent of code coverage, as well as broaden the types of applications that can be closely examined.  Firstly, it would be desireable to add a static analysis phase to \toolname~to allow the enumeration of API calls made by the application.  VM instrumentation, the ActivityManager's profiling functionality, or application rewriting could then be used to determine when an API call is executed.  Secondly, \toolname~does not handle applications that depend on external third-party libraries such as Adobe Air.  Additional analysis of the dataset will be required to determine the prevalence of these libraries so they can be included in the emulator image.  

While testing the applications, it was observed that third-party libraries included in the APK tend to balloon the permission requirements far beyond what the applications themselves required. For example, the flashlight application mentioned earlier has most of it's permissions primarily because they are required for the numerous ad libraries it uses to serve advertisements to the user. In the future, mapping common libraries to the permissions they require, or even these library calls to permissions, as in~\cite{Felt}, would be desireable. Lastly, applications that use intent receivers guarded by permissions to protect message passing are not handled; ultimately, more research needs done to determine if handling this case is necessary.  It should be noted, however, that activities intended to be invoked by these receivers are already being executed.

Overall, seven different permissions that have a high security or privacy impact were successfully tested but the surface has only been scratched. In the future, the usage of many other permissions, with an initial focus on those that pose a security risk (e.g.billing), should be investigated. Furthermore, the removal of various permission combinations, such as \texttt{ACCESS\_FINE\_LOCATION} and \texttt{ACCESS\_COARSE\_LOCATION}, should be tested.  
